\section{Financial Statement Analysis}

\subsection{Market Value Metrics}
Market Capitalisation is the total market value of a firm's equity based off the \textbf{Market Price}.

$\text{Market Cap} = \text{Shares Outstanding} \times \text{Market Price}$

Market Value Added is the difference between the market value of the firm's equity and the book value of the firm's equity.

$\text{MVA} = \text{Market Cap} - \text{Book Value of Equity}$

Market to Book Ratio is the ratio of the market value of the firm's equity to the book value of the firm's equity.

$\text{Market to Book Ratio} = \frac{\text{Market Cap}}{\text{Book Value of Equity}}$

Issues with Market Value Metrics:
\begin{itemize}
    \item Market value is based on expectations of future performance.
    \item Market value is based on the market price, which may not reflect the true value of the firm.
    \item Market value is based on the market price, which may be influenced by market sentiment.
    \item Market value cannot be found for private firms.
\end{itemize}

Economic Value Added (EVA) is the difference between the firm's net operating
profit after taxes and the cost of capital.

$EVA = \text{ATOI} - \text{Cost of Capital}$

After Tax Operating Income (ATOI) is the firm's operating profit after taxes. 

ATOI = EBIT(1-$T_C$)

\begin{itemize}
    \item EVA is a measure of the firm's economic profit.
    \item EVA is a measure of the firm's value creation.
    \item EVA is a measure of the firm's ability to generate returns above the cost of capital.
\end{itemize}

\subsection{Accounting Rate of Return}

Return on Capital (ROC) is the ratio of Net Income plus after tax interest to long-term debt.

$\text{ROC} = \frac{EBIT\left(1-T_C\right)}{\text{Long-Term Capital}}$

Return on Assets (ROA) is the ratio of Net Income plus after tax interest to Total Assets.

$\text{ROA} = \frac{EBIT\left(1-T_C\right)}{\text{Total Assets}}$

Return on Equity (ROE) is the ratio of Net Income plus after tax interest to Total Equity.

$\text{ROE} = \frac{EBIT\left(1-T_C\right)}{\text{Total Equity}}$

Difference between ROE, ROC and ROA:
\begin{itemize}
    \item ROE is a measure of the return to the firm's equity holders.
    \item ROC is a measure of the return to the firm's long-term debt holders.
    \item ROA is a measure of the return to the firm's total assets.
\end{itemize}

Asset Turnover Ratio is the ratio of Sales to Total Assets.

$\text{Asset Turnover Ratio} = \frac{\text{Sales}}{\text{Total Assets at start of year}}$

Inventory Turnover Ratio is the ratio of Cost of Goods Sold to Inventory.

$\text{Inventory Turnover Ratio} = \frac{\text{Cost of Goods Sold}}{\text{Inventory at start of year}}$

Average days in inventory is how long inventory is held before being sold.

$\text{Average Days in Inventory} = \frac{365}{\text{Inventory Turnover Ratio}} = \frac{\text{Inventory at start of year}}{\text{Cost of Goods Sold}/ 365}$

Receivables Turnover Ratio is the ratio of Sales to Accounts Receivable.

$\text{Receivables Turnover Ratio} = \frac{\text{Sales}}{\text{Accounts Receivable at start of year}}$

Average collection period is how long it takes to collect receivables.

$\text{Average Collection Period} = \frac{365}{\text{Receivables Turnover Ratio}} = \frac{\text{Accounts Receivable at start of year}}{\text{Sales}/ 365}$

\subsection{DuPont Analysis}

Profit Margin is the ratio of Net Income to Sales.

$\text{Profit Margin} = \frac{\text{Net Income}}{\text{Sales}}$

Operating Profit Margin is the ratio of EBIAT to Sales.

$\text{Operating Profit Margin} = \frac{\text{EBIT}(1-T_C)}{\text{Sales}}$

ROA is After Tax Operating Income (AKA $EBIT(1-T_C)$ to Total Assets.

$\text{ROA} = \frac{EBIT(1-T_C)}{\text{Total Assets}} = \frac{Sales}{Assets}\times \frac{ATOI}{Sales}$

When merging with another company, companies may find that unless they have some special skill
in running the acquired company, the ROA of the merged company will be the same as the ROA of the acquiring company. As the gain in Profit Margin is offset by a decline in Asset Turnover.


\subsection{Financial Leverage}

Long Term Debt Ratio is the ratio of Long Term Debt to Total Assets.

$\text{Long Term Debt Ratio} = \frac{\text{Long Term Debt}}{\text{Long Term Debt + Equity} } = \frac{\text{Long Term Debt}}{\text{Total Assets - Current Liabilities}}$

Long Term Debt to Equity Ratio is the ratio of Long Term Debt to Total Equity.

$\text{Long Term Debt to Equity Ratio} = \frac{\text{Long Term Debt}}{\text{Total Equity}}$

Total Debt Ratio is the ratio of Total Debt to Total Assets.

$\text{Total Debt Ratio} = \frac{\text{Total Debt}}{\text{Total Assets}}$

Times Interest Earned is the ratio of EBIT to Interest Expense.

$\text{Times Interest Earned} = \frac{\text{EBIT}}{\text{Interest Expense}}$

Cash Coverage Ratio is the ratio of EBIT plus Depreciation to Interest Expense.

$\text{Cash Coverage Ratio} = \frac{\text{EBIT} + \text{Depreciation}}{\text{Interest Expense}}$

$ROE = \frac{Assets}{Equity} \times \frac{Sales}{Assets} \times \frac{EBIT(1-T_C)}{Sales} \times \frac{Net Income}{EBIT(1-T_C)} =  \text{Leverage Ratio} \times \text{Asset Turnover} \times \text{Operating Profit Margin} \times \text{Debt Burden}$


\subsection{Measuring Liquidity}

Net Working Capital is the difference between Current Assets and Current Liabilities.

$\text{Net Working Capital} = \text{Current Assets} - \text{Current Liabilities}$

Net Working Capital to Total Assets Ratio is the ratio of Net Working Capital to Total Assets.

$\text{Net Working Capital to Total Assets Ratio} = \frac{\text{Net Working Capital}}{\text{Total Assets}}$

Current Ratio is the ratio of Current Assets to Current Liabilities.

$\text{Current Ratio} = \frac{\text{Current Assets}}{\text{Current Liabilities}}$

Quick Ratio is the ratio of Current Assets minus Inventory to Current Liabilities.

$\text{Quick Ratio} = \frac{\text{Current Assets} - \text{Inventory}}{\text{Current Liabilities}}$

Cash Ratio is the ratio of Cash and Cash Equivilants to Current Liabilities.

$\text{Cash Ratio} = \frac{\text{Cash} + \text{Cash Equivilants}}{\text{Current Liabilities}}$


\subsection{Miscellaneous Ratios}
Earnings Per Share (EPS) is the ratio of Net Income to Shares Outstanding.

$\text{EPS} = \frac{\text{Net Income}}{\text{Shares Outstanding}}$

Share Price is the ratio of Market Capitalisation (i.e. Market Value of Equity) to Shares Outstanding.

$\text{Share Price} = \frac{\text{Market Cap}}{\text{Shares Outstanding}}$
