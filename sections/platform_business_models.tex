\section{Platform Business Models}
The traditional system of value creation is linear, with value created 
by producers and consumed by customers. In contrast, platform business models 
create value by facilitating interactions between producers and consumers. 

Platforms can be classified into three types: innovation platforms, transaction platforms, and hybrid platforms. 

The key to success in platform business models is to create a virtuous 
cycle of value creation,
where the value of the platform increases as more users join the platform. 

This can be achieved by leveraging network effects, data-driven insights, and ecosystem orchestration.

\begin{itemize}
	\item No gatekeepers: Platforms allow producers and consumers to interact directly, without intermediaries. Gatekeepers are replaced by market signals.
	\item New sources of value and supply, by not owning the assets, expansion is much less risky.
	\item Community feedback loops: Platforms can leverage the wisdom of the crowd to improve the quality of products and services.
\end{itemize}

\subsection{Network Effects}
It is the impact where the number of users of a platform increases the value of the platform created for each user.

\begin{itemize}
	\item Direct network effects: The value of the platform increases as more users join the platform.
	\item Indirect network effects: The value of the platform increases as more complementary products and services are offered on the platform.
\end{itemize}

The sides of a platform may be different and can have different incentives to join the network.

\begin{callout}
	\textbf{Example:} In the case of Uber, drivers are incentivized to join the platform by the promise of higher earnings, while riders are incentivized to join the platform by the promise of lower prices.

	Virality can attract users, but the network effects are what keep them on the platform.
\end{callout}


Network effects can be positive or negative. Positive network effects occur when the value of 
the platform increases as more users join the platform. 
Negative network effects occur when the value of the platform decreases as more users join the platform.

\begin{callout}
	\textbf{Example:} OkCupid, a dating platform, experienced negative network effects when it began to scale up, and it was harder to match users, driving them away to other platforms.
\end{callout}

\begin{itemize}
	Types of Network Effects:
	\item Positive Same-Side: The positive benefits received by users of the same type as more users of the same type join the platform.
	\item Negative Same-Side: The negative effects experienced by users of the same type as more users of the same type join the platform, such as competition.
	\item Positive Cross-Side: The positive benefits received by users of one type as more users of another type join the platform.
	\item Negative Cross-Side: The negative effects experienced by users of one type as more users of another type join the platform.
\end{itemize}

\subsection{Monetization Strategies}
There are several ways to monetize a platform business model:

\begin{itemize}
	\item Transaction fees: Platforms can charge a fee for each transaction that takes place on the platform.
	\item Subscription fees: Platforms can charge a subscription fee for access to the platform.
	\item Advertising: Platforms can generate revenue by selling advertising space on the platform.
	\item Data monetization: Platforms can monetize the data generated by users on the platform.
	\item Freemium: Platforms can offer a basic service for free and charge for premium features.
\end{itemize}

To take note, charging for usage can be a barrier to entry, and it can be difficult to change the monetization strategy once users are accustomed to it.

Possible way to do it is to charge users in order to reduce negative network effects. 
Such as with dating apps, charging for the app reduces users and makes them easuer to match.

Another way is to charge for the service that is most valuable to the user, such as charging recruiters on LinkedIn to present job opportunities.

In the pipeline mode, common metrics to measure efficiency are
Cash Flow, Inventory Turnover, Operating Income, and Return on Assets etc.

In the platform mode, common metrics to measure efficiency are
Customer Acquisition Cost, Customer Lifetime Value, Network Effects, and Churn Rate etc.


