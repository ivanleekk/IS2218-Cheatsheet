\section{Accounting}
\subsection{Balance Sheet}
The balance sheet is a snapshot of the firm's assets and liabilities at a given point in time.

At all times, $Assets = Liabilities + Equity$

Current Assets/Liabilities are expected to be converted to cash or paid \textbf{within one year}.

Long-term Assets/Liabilities are expected to be converted to cash or paid \textbf{after one year}.

Book Value is the value of the firm's assets as shown on the balance sheet.

However, the market value of the firm's assets may be different from the book value, and is determined
by the price at which the assets could be bought or sold on financial markets.

\subsection{Income Statement}
The income statement shows the firm's revenues and expenses over a period of time.

Earnings Before Interest and Taxes (EBIT) is the firm's operating profit.
$EBIT = Revenue - Other Income - Costs - Depreciation$

Net Income is the firm's profit after additional expenses such as interest and taxes.
$Net Income = EBIT - Interest - Taxes$

Profits ignore cash expenditures such as capital expenditures at the point they are made, but only realises them 
as depreciation over time.

Profits also are recorded at the time of sale, and \textbf{not} when the cash is received.

\subsection{Cash Flow Statement}
The cash flow statement shows the firm's cash inflows and outflows over a period of time.

\textbf{Cash Flow from Operations (CFFO)} is the cash generated by the firm's normal business operations.

\textbf{Cash Flow from Investing (CFFI)} is the cash generated by the firm's investments in long-term assets.

\textbf{Cash Flow from Financing (CFFF)} is the cash generated by the firm's financing activities.

\textbf{Free Cash Flow} is the cash generated by the firm's operations after accounting for capital expenditures, but before taxes.
$FCF = Net Income + Interest + Depreciation - Additions to Net Working Capital + CFFI = Interest + CFFO + CFFI$

\subsection{Tax Rates}

\begin{tabularx}{\linewidth}{X X X}
\toprule
\multicolumn{3}{c}{\textbf{Taxable Income (dollars)}} \\
\midrule
\textbf{Single Taxpayers} & \textbf{Married Taxpayers Filing Joint Returns} & \textbf{Tax Rate} \\
\midrule
0 - 9,525 & 0 - 19,050 & 10.0\% \\
9,525 - 38,700 & 19,050 - 77,400 & 12.0\% \\
38,700 - 82,500 & 77,400 - 165,000 & 22.0\% \\
82,500 - 157,500 & 165,000 - 315,000 & 24.0\% \\
157,500 - 200,000 & 315,000 - 400,000 & 32.0\% \\
200,000 - 500,000 & 400,000 - 600,000 & 35.0\% \\
500,000 and above & 600,000 and above & 37.0\% \\
\toprule
\multicolumn{3}{c}{\textbf{Corporate Taxes}} \\
\midrule
\textbf{Taxable Income (dollars)} & \textbf{Tax Rate} \\
\midrule
All & 21.0\% \\
\bottomrule
\end{tabularx}
