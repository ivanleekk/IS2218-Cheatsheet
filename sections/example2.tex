\section{Example 2, Title}
Financial Institutions are intermidiaries that channel funds 
from savers to borrowers

They provide economies of scale and scope, risk sharing and alleviate
problems with assymetric information, but also could have conflicts of interest

\subsection{Depository Financial Institutions}
Accept deposits from individuals and institutions and provide loans
% \begin{itemize}
%     \item Commercial Banks
%     \item Savings Banks
%     \item Finance Companies (SG)
%     \item Savings and Loans Associations (US)
%     \item Credit Unions (US)
% \end{itemize}

Commercial Banks accept deposits and provide loans

They can provide Checking Accounts and Savings Accounts (CASA), Time Deposits, and Certificates of Deposit etc.

Corporate loans, Consumer loans, and Mortgages are also provided

They provide payment services, transfers and some also have wealth management services

Commercial Banks profit from interest margins, fees, commissions and off-balance sheet activities

\begin{callout}
    Full Banks in SG can conduct universal banking, while Wholesale Banks can only conduct limited SGD retail banking
\end{callout}

Finance Companies primarily focus on offering loans, credit facilities and deposit services to individuals and SMEs

They do not have checking accounts and have restricted dealings in foreign currency, gold and other precious metals

There are also limits on (Loans per borrower $\leq$ 0.5\% of capital funds) and (Total unsecured loans $\leq$ 25\% of capital funds)

\subsection{Non-Depository Financial Institutions}
Do not accept deposits but provide financial services, such as insurance, investment banking, brokerage, and asset management

Insurance Companies transfer risk to itself in exchange for a premium, pooling risk together

Investing these premium income to earn returns, and paying out claims when necessary

However, since the timing and magnitude of claims are uncertain, they need to have a large pool of policyholders to spread the risk


Investment Banks provide services such as Underwriting, Mergers and Acquisitions, Asset Securitisation and the Creation of Financial Products

Underwriting helps corporations raise capital by issuing securities to the general public for the first time (IPO) or by selling to selected investors (Private Placement)

Mergers and Acquisitions help companies find suitable targets for acquisition or merger, or potential buyers

Investment banks create financial products such as derivatives, structured products, and securitised products by 
writing contracts that derive their value from an underlying asset

Asset Securitisation is the process of converting illiquid assets into liquid assets by pooling them together and issuing securities backed by these assets, such as mortgage-backed securities (MBS)


Investment Companies pool funds from investors to invest, with different types of funds such as Mutual Funds, Exchange-Traded Funds (ETFs), and Hedge Funds for different risk appetites and strategies

Open-End Mutual Funds issue and redeem shares at the Net Asset Value (NAV) of the fund, and are priced at the end of the trading day

Closed-End Mutual Funds issue a fixed number of shares that trade on an exchange, and are priced by supply and demand, with NO money flow in and out of the funds between inception and termination
\begin{callout}
    The fact that these closed-end funds trade at a premium or dicount to NAV is the Closed End Fund Puzzle
\end{callout}

Unit Trusts (US) are basically \textbf{unmanaged} closed-end mutual funds and a fixed termination date

UCITS is Eurpoean mutual funds that can be sold across the EU

Hedge Funds are partnerships that operate through private placement to rich individuals and institutions, shorting overvalued assets and longing undervalued assets (hedging), and using leverage to amplify returns

Private Equity Funds are funds that invest in private companies, with the aim of improving the company's performance and selling it at a profit
With a General Partner that holds controlling power and Limited Partners that provide funds and act as investors in the fund

Pension Funds are funds that provide retirement income to employees, with Defined Benefit Plans that promise a fixed income and Defined Contribution Plans that depend on the performance of the fund.
Portability refers to if the employee can bring the pension fund to the next employer.
\begin{callout}
    In SG, CPF is a Defined Contribution Plan that is Portable and compulsory for all citizens and PRs
\end{callout}

Soverign Wealth Funds are state-owned investment funds, that take funds from exports or taxes and invest them to generate returns. Total AUM is about \$12 trillion in Ju;y 2024

