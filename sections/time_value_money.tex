\section{Time Value of Money}
\subsection{Simple Interest}
\begin{align*}
	FV &= PV(1 + rt) \\
	PV &= \frac{FV}{1 + rt} \\
	r &= \frac{FV - PV}{PVt} \\
	t &= \frac{FV - PV}{PVr}
\end{align*}
\subsection{Compound Interest}
\begin{align*}
	FV &= PV(1 + r)^t \\
	PV &= \frac{FV}{(1 + r)^t} \\
	r &= \left(\frac{FV}{PV}\right)^{\frac{1}{t}} - 1 \\
	t &= \frac{\ln\left(\frac{FV}{PV}\right)}{\ln(1 + r)}\\
	\text{Discount Factor} DF &= \frac{1}{(1 + r)^t}
\end{align*}
Discount factor is the PV of \$1 to be received in $t$ years.
\subsection{Effective Interest Rate}
\begin{align*}
	r_{\text{eff}} &= \left(1 + \frac{r_{\text{nom}}}{n}\right)^n - 1\\
	r_{\text{nom}} &= n\left(\left(1 + r_{\text{eff}}\right)^{\frac{1}{n}} - 1\right)
\end{align*}
Effective interest rate is the annual rate that reflects the effects of compounding.
\begin{callout}
	\textbf{Example:} If the APR (nominal rate) is 6\% compounded quarterly, the effective rate is $(1 + \frac{0.06}{4})^4 - 1 = 6.136\%$.
\end{callout}
\subsection{Annuities}
\begin{align*}
	PV &= \frac{PMT}{r} \left(1 - \frac{1}{(1 + r)^t}\right) \\
	   &= PMT \left(\frac{1}{r} - \frac{1}{r\cdot (1 + r)^t}\right) \\
	FV &= PMT \left(\frac{(1 + r)^t - 1}{r}\right) \\
	PMT &= \frac{PV\cdot r}{1 - \frac{1}{(1 + r)^t}} \\
	t &= \frac{\ln\left(\frac{FV}{PMT}\right)}{\ln(1 + r)} \\
	r &= \frac{PMT}{PV} \left(1 - \frac{1}{(1 + r)^t}\right)\\
PVAF &= \frac{1}{r} - \frac{1}{r\cdot (1 + r)^t}
\end{align*}
Annuity is a series of equal payments made at regular intervals with a set maturity.
\subsubsection{Annuity Due}
\begin{align*}
	PV_{\text{annuity due}} &= \frac{PMT}{r} \left(1 - \frac{1}{(1 + r)^t}\right) \cdot (1 + r) \\
	&= PMT \left(\frac{1}{r} - \frac{1}{r\cdot (1 + r)^t}\right) \cdot (1 + r) \\
	FV_{\text{annuity due}} &= PMT \left(\frac{(1 + r)^t - 1}{r}\right) \cdot (1 + r) \\
\end{align*}
Annity due is an annuity where payments are made at the beginning of each period instead of the end.
\subsection{Perpetuities}
\begin{align*}
	PV &= \frac{PMT}{r} \\
	PMT &= PV\cdot r
\end{align*}
Perpetuity is a series of equal payments made at regular intervals with no maturity.
\subsection{Amortization}
\begin{align*}
	PMT &= \frac{PV\cdot r}{1 - \frac{1}{(1 + r)^t}} \\
	\text{Interest Payment}_t &= PV_t\cdot r \\
	\text{Principal Payment}_t &= PMT - \text{Interest Payment}_t\\
	\text{Remaining Principal}_t &= \frac{PMT}{r} \left(1 - \frac{1}{(1 + r)^{n-t}}\right)
\end{align*}
\subsection{Inflation}
\begin{align*}
	1 + \text{Real Rate} &= \frac{1 + \text{Nominal Rate}}{1 + \text{Inflation Rate}} \\
	\text{Real Rate} &\approx \text{Nominal Rate} - \text{Inflation Rate}
\end{align*}
